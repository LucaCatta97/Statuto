\documentclass[legalpaper, 11pt]{exam}


\usepackage{hyperref}
\usepackage{lineno}

\hypersetup
{
  colorlinks,
  citecolor=black,
  linkcolor=black,
  urlcolor=black
}

\usepackage{color,soul}

\usepackage{titlesec}
\titlespacing*{\section}{0pt}{0.5\baselineskip}{0.5\baselineskip}

\tolerance=1
\emergencystretch=\maxdimen
\hyphenpenalty=10000
\hbadness=10000

\usepackage[utf8]{inputenc}
\usepackage[left=1.5cm,
            right=1.5cm,
            top=1.5cm,
            bottom=1.5cm]{geometry}
\usepackage{lastpage}

\cfoot{}
\rfoot{}
\rhead{}

\let\tempone\enumerate
\let\temptwo\endenumerate
\renewenvironment{enumerate}{\tempone\addtolength{\itemsep}{-0.45\baselineskip}}{\temptwo}

%\usepackage{fontspec}
%\defaultfontfeatures{Mapping=tex-text}
%\setmainfont{Verdana}


%\graphicspath{{assets/}}

%\iffalse
\usepackage{fontspec}
\setmainfont{Ubuntu} [
    Path = fonts/Ubuntu/,
    Extension = .ttf,
    UprightFont = *-Regular,
    ItalicFont = *-Italic,
    BoldFont = *-Bold,
    BoldItalicFont = *-BoldItalic
]
%\fi


%\title{PoliNetwork \\ Statuto}
%\author{PoliNetwork}
%\date{\today}


%\addto\captionsenglish{% 
%  \renewcommand{\contentsname}%
%    {Indice}%
%}

\usepackage{setspace}


\begin{document}

%\linenumbers

{
\setstretch{0.5}

\begin{center}


\begin{center}
{\textbf{Statuto PoliNetwork APS}}

\end{center}



\end{center}


\section{Denominazione}
\begin{enumerate}
 \item È costituito, nel rispetto del Codice civile, del D. Lgs. 117/2017 e della normativa in materia, l’Ente del Terzo Settore denominato: associazione “PoliNetwork APS”, di seguito denominata “PoliNetwork”.
 \item PoliNetwork assume la forma giuridica di associazione non riconosciuta, apartitica e aconfessionale.
 \item L'associazione ha durata illimitata.
 \item Poiché la qualificazione giuridica di Associazione di Promozione Sociale discende, tra l'altro, dall'iscrizione nel registro unico nazionale del Terzo settore, l'acronimo APS, anche se previsto nella denominazione sociale, non sarà spendibile nei rapporti con i terzi, negli atti, nella corrispondenza e nelle comunicazioni con il pubblico sino al perfezionamento della procedura di iscrizione al competente RUNTS.
\end{enumerate}

\section{Definizioni}
\begin{enumerate}
 \item Si definisce PoliNetwork anche come "associazione" oppure "organizzazione".
 \item Si definisce ”limite temporale” il trascorrere di 30 (trenta) giorni, 90 (novanta) giorni se la richiesta è presentata nei mesi di Gennaio, Febbraio, Giugno, Luglio, Agosto e Settembre.
 \item Si definisce ”canale di contatto” qualsiasi mezzo di comunicazione scritta definito nel regolamento interno dell’associazione e utilizzato per qualsiasi richiesta a cui si fa riferimento nel presente Statuto, purché gestito nel rispetto della normativa nazionale e comunitaria in materia di protezione dei dati personali.
\end{enumerate}

\section{Sede}
\begin{enumerate}
 \item L’associazione ha sede legale all’interno del comune di Milano.
 \item Il trasferimento della sede legale non comporta modifica statutaria, ma l’obbligo di comunicazione agli uffici competenti.
\end{enumerate}

\section{Statuto}
\begin{enumerate}
 \item L’associazione è disciplinata dal presente Statuto, ed agisce nei limiti del D. Lgs. 117/2017, delle relative norme di attuazione, della legge regionale e dei principi generali dell’ordinamento giuridico.
 \item Lo Statuto vincola alla sua osservanza gli aderenti all’associazione; esso costituisce le regole fondamentali di comportamento e l’organizzazione della stessa.
\end{enumerate}

\section{Scopo, finalità e attività}
\vspace{-5pt}
\begin{enumerate}
 \item L’Associazione esercita in via esclusiva o principale una o più attività di interesse generale per il perseguimento, senza scopo di lucro, di finalità civiche, solidaristiche e di utilità sociale.
 \item Ai sensi dell’art 5 D.Lgs n.117/2017 la/e attività che si propone di svolgere avvalendosi in modo prevalente dell’attività di volontariato dei propri Associati è/sono:
 \begin{enumerate}
	\item istruzione e formazione sociale;
	\item formazione universitaria;
	\item organizzazione e gestione di attivita' culturali, artistiche o ricreative di interesse sociale, incluse attività, anche editoriali, di promozione e diffusione della cultura e della pratica	del volontariato e delle attivita' di interesse generale di cui al presente articolo;
 \end{enumerate}
 \item Ai fini della realizzazione del comma precedente l’associazione intende intraprendere le seguenti azioni:
 \begin{enumerate}
	\item organizza e gestisce eventi formativi, culturali e di aggregazione;
	\item promuove la comunicazione fra gli studenti del Politecnico di Milano attraverso le tecnologie	di comunicazione ritenute più idonee;
	\item promuove l’interazione e la cooperazione fra studenti universitari e altri enti;
	\item ogni altra attività utile al raggiungimento degli scopi sociali.
 \end{enumerate}
 \item L’Associazione persegue i seguenti principi:
 \begin{enumerate}
	\item è libera da ogni condizionamento politico, sociale, economico e religioso;
	\item I progetti di carattere tecnico informatico dovranno avere licenza open source.
 \end{enumerate}
\end{enumerate}

\section{Organi}
Sono organi dell’Associazione: 
\vspace{-5pt}
\begin{enumerate}
 \item il Consiglio Direttivo (di seguito anche Direttivo);
 \item le Assemblee dei Soci ordinaria e straordinaria;
 \item Il Collegio dei Probiviri (di seguito anche Collegio).
\end{enumerate}

\section{Associati}
\begin{enumerate}
 \item Il numero dei Soci è illimitato. 
 \item Gli aspiranti soci sono ammessi senza alcuna forma di discriminazione di sesso, identità di genere, etnia, opinione politica, orientamento sessuale e religione.
 \item Essi devono condividere gli scopi dell’associazione.
\end{enumerate}

\section{Volontari}
\begin{enumerate}
 \item Le attività perpetuate dai Soci ai fini del raggiungimento degli scopi associativi sono svolte puramente a scopo di volontariato e non condizionate ad alcun compenso.
 \item L’associazione può avvalersi nello svolgimento delle proprie attività dell’opera di volontari. Sono volontari coloro che per libera scelta, senza divenire associati, svolgono attività in favore di PoliNetwork o dei suoi progetti mettendo a disposizione il proprio tempo e le proprie capacità in modo personale e gratuito senza fini di lucro. I volontari possono essere iscritti in apposito Registro dei Volontari.
\end{enumerate}

\section{Ammissione}
\begin{enumerate}
 \item Chi intende essere ammesso come Socio dovrà presentare la relativa richiesta al Consiglio Direttivo, impegnandosi ad attenersi al presente Statuto, ad osservarne gli eventuali regolamenti interni e le delibere adottate dagli organi dell’associazione e a mantenere attivo il canale di contatto per tutta la durata del rapporto associativo. Il Consiglio Direttivo, o disgiuntamente ogni suo componente o delegato, potrà chiedere all’aspirante Socio ogni documentazione utile ed avvalersi di qualsiasi mezzo ritenga necessario al fine di valutare la richiesta di ammissione, nel rispetto della normativa vigente, sia comunitaria sia nazionale, anche in tema di tutela dei dati personali e della privacy.
 \item Il Consiglio Direttivo deciderà, entro il limite temporale, sull’esito positivo o meno della richiesta.
 \item La delibera di rigetto della domanda di ammissione dovrà essere motivata all’Assemblea dei Soci e all’aspirante Socio entro sette giorni dall’adozione. L’aspirante socio può chiedere all’Assemblea di riesaminare la decisione entro il limite temporale decorrente dalla comunicazione della delibera di rigetto. l’Assemblea si esprime definitivamente nella seduta immediatamente successiva.
 \item All’atto dell’accettazione della richiesta da parte dell’associazione il Segretario dovrà comunicare al richiedente la decisione entro dieci giorni. Il richiedente dovrà versare la quota associativa (nelle modalità specificate nel regolamento interno) e successivamente acquisirà ad ogni effetto la qualifica di Socio con obbligo per il Segretario di annotarlo nel registro dei soci entro sette giorni dal pagamento. Entro sette giorni dall’iscrizione al libro dei Soci il Segretario trasmette conferma di iscrizione con relativo codice di iscrizione.
 \item Tutti i Soci verranno annotati nel libro dei Soci dell’associazione.
\end{enumerate}

\section{Diritti e doveri degli associati}
\begin{enumerate}
 \item Gli Associati dell'organizzazione hanno il diritto di:
 \vspace{-5pt}
 \begin{enumerate}
  \item eleggere gli organi sociali e di essere eletti negli stessi, secondo il presente Statuto;
  \item essere informati sulle attività dell’organizzazione e controllarne l’andamento;
  \item prendere atto dell’ordine del giorno delle assemblee, prendere visione del rendiconto
  economico-finanziario, consultare i verbali;
  \item votare in Assemblea. Ciascun associato ha diritto al più ad un voto.
 \end{enumerate}
\item Gli stessi Associati hanno il dovere di:
\vspace{-5pt}
\begin{enumerate}
 \item rispettare il presente Statuto e il regolamento interno;
 \item astenersi da qualsiasi atto che possa nuocere all’organizzazione;
 \item versare la quota associativa secondo l’importo stabilito ove previsto dal regolamento interno. La quota associativa è personale, non è rimborsabile e non può essere trasferita a terzi o rivalutata.
\end{enumerate}

\end{enumerate}

\section{Recesso}
\begin{enumerate}
 \item La qualità di Socio si perde per:
 \vspace{-5pt}
 \begin{enumerate}
  \item mancato pagamento della quota sociale, nelle casistiche definite dal regolamento interno: la decadenza avviene su decisione dell’Assemblea, previa proposta del Consiglio Direttivo, trascorso un mese dal mancato versamento della quota sociale;
  \item dimissioni: ogni Socio può recedere dall’associazione in qualsiasi momento dandone
  comunicazione scritta al Consiglio Direttivo che invierà la stessa in copia conoscenza al consiglio dei probiviri; tale recesso avrà decorrenza immediata. Resta fermo l’obbligo di pagamento della quota sociale per l’anno in corso ove prevista dal regolamento interno;
  \item espulsione: il Consiglio Direttivo delibera l’espulsione del Socio, previa contestazione degli eventuali addebiti e, se possibile, sentito il Socio interessato, per atti compiuti in contrasto a quanto previsto dal presente statuto o dal regolamento interno o qualora siano intervenuti gravi motivi che rendano incompatibile la prosecuzione del rapporto associativo. Il Consiglio Direttivo comunica la decisione all’Assemblea, che ha facoltà di opporsi alla decisione, sempre entro il limite temporale.
  \item decesso.
 \end{enumerate}

 \item Gli Associati che abbiano comunque cessato di appartenere all’associazione non possono richiedere i contributi versati e non hanno alcun diritto sul patrimonio dell’associazione stessa.
 \item Chiunque perda o non abbia mai ottenuto la qualità di Socio e abbia dato un contributo all’associazione può richiedere al Direttivo di essere inserito nel libro dei soci emeriti di PoliNetwork. Se entro il limite temporale non è stata ricevuta risposta la richiesta è da intendersi respinta. I fondatori sono inseriti di diritto. Le disposizioni del presente comma si applicano anche a coloro che abbiano partecipato alle attività di PoliNetwork prima dell’emanazione del presente statuto. L’Assemblea può in qualsiasi momento rimuovere chiunque dal libro dei soci emeriti.
\end{enumerate}

\section{Assemblea}
\begin{enumerate}
 \item L’Assemblea è l’organo sovrano di PoliNetwork ed è composta da tutti i Soci dell’organizzazione.
 \item L’Assemblea è convocata e presieduta dal Presidente del Direttivo.
 \item L’Assemblea si riunisce di diritto almeno due all’anno.
 \item La convocazione deve essere inviata mediante canale di contatto entro sette giorni prima della riunione indicando orario, luogo (anche in videoconferenza), ordine del giorno e all’orario delle eventuali ulteriori convocazioni in caso di mancato raggiungimento dei quorum costitutivi.
 %\item L’Assemblea viene convocata per l’approvazione del bilancio o del rendiconto e quando ne è fatta richiesta motivata da almeno un quinto dei Soci o quando il Collegio dei Probiviri o il Consiglio Direttivo o disgiuntamente ogni suo componente o delegato lo ritiene necessario.
 \item Delle riunioni dell’Assemblea è redatto il verbale, sottoscritto dal Presidente e dal verbalizzante.
 \item L’Assemblea ordinaria delibera su tutti e soli i punti elencati al comma successivo del presente Statuto. L’Assemblea straordinaria viene proposta ogni qual volta sia necessario per le esigenze dell’associazione.
 \item L'Assemblea ordinaria:
 \vspace{-5pt}
 \begin{enumerate}
  \item approva il bilancio come previsto dal successivo articolo 19 del presente Statuto;
  \item determina le linee generali programmatiche riguardo l’attività dell’organizzazione;
  \item elegge i componenti del Consiglio Direttivo. In tal caso è presieduta dal Presidente della Commissione Elettorale;
  \item revoca i componenti del Consiglio Direttivo, o le singole cariche, su proposta di almeno un terzo degli associati. In tal caso è presieduta e convocata dal Socio più anziano per iscrizione e in caso di parità per età, il quale designa un verbalizzante fra i presenti;
  \item elegge e revoca, quando previsto, il Collegio dei Probiviri nel rispetto dello specifico regolamento di funzionamento;
  \item delibera la responsabilità dei componenti degli organi sociali e promuove azione di responsabilità nei loro confronti;
  \item approva gli eventuali regolamenti interni proposti dagli associati;
  \item ratifica, quando richiesto da almeno un socio, i regolamenti emanati dal Direttivo;
  \item può opporsi all’esclusione dei soci nelle modalità previste dallo Statuto;
  \item su ogni altra materia per la quale la normativa vigente o il presente Statuto richiedano la deliberazione dell'assemblea ordinaria.
 \end{enumerate}
 \item L'Assemblea straordinaria:
 \vspace{-5pt}
 \begin{enumerate}
  \item delibera sulle modifiche dello Statuto dell’organizzazione proposte dal Direttivo o dagli associati in un numero non inferiore al terzo degli iscritti;
  \item delibera la proroga del mandato del Direttivo;
  \item delibera lo scioglimento con conseguente liquidazione nonché devoluzione del patrimonio e nomina dei liquidatori;
  \item delibera sulla fusione, trasformazione dell’Associazione;
  \item delibera su ogni altra materia per la quale la normativa vigente o il presente Statuto richiedano la deliberazione dell'assemblea straordinaria.
 \end{enumerate}
 \item Le Assemblee ordinarie sono regolarmente costituite in prima convocazione con la presenza della maggioranza assoluta degli aderenti, presenti in proprio o per delega, e in seconda convocazione, da tenersi anche nello stesso giorno, qualunque sia il numero degli aderenti presenti, in proprio o per delega.
 \item Le Assemblee straordinarie sono regolarmente costituite in prima convocazione con la presenza di più dei due terzi degli aderenti, presenti in proprio, e in seconda convocazione, da tenersi non prima di una settimana dalla prima convocazione, purché siano presenti la maggioranza assoluta degli iscritti, in proprio o per delega, e in terza convocazione, da tenersi non prima di tre giorni dalla seconda convocazione, qualsiasi sia il numero di iscritti presenti in proprio o per delega.
 \item L’Assemblea ordinaria delibera a maggioranza dei voti dei presenti, in caso di parità prevale il voto del Presidente tranne nel caso di convocazione per elezione o revoca dei componenti del Direttivo.
 \item L’Assemblea straordinaria delibera a maggioranza dei due terzi dei voti dei presenti.
 \item Gli Associati possono farsi rappresentare in Assemblea solo da altri Associati, conferendo delega scritta tramite canale di contatto. Ciascun Associato è portatore di al più tre deleghe. La delega, come previsto nel comma 8 del presente articolo , non è ammessa in sede di prima convocazione di un’assemblea straordinaria. Non è possibile delegare membri del Direttivo.
 \item Le delibere sono espresse con voto palese, fatta eccezione per quelle riguardanti singoli o ristretti gruppi di individui o quando l’Assemblea, il Direttivo o il Socio firmatario della proposta lo ritenga opportuno o se prescritto dal regolamento interno, le quali sono espresse con voto segreto;
\end{enumerate}

\section{Direttivo}
\begin{enumerate}
 \item Il Direttivo è composto da un minimo di 5 (cinque) a un massimo di 9 (nove) persone.
 \item Il Consiglio Direttivo compie tutti gli atti di ordinaria e straordinaria amministrazione la cui competenza non sia, per norma di legge, di pertinenza esclusiva dell’Assemblea. In particolare, tra gli altri compiti:
\begin{enumerate}
\item amministra l’Associazione;
\item attua le deliberazioni dell’Assemblea;
\item predispone il bilancio come previsto dal successivo articolo 19 del presente Statuto;
\item stipula tutti gli atti e contratti inerenti le attività associative;
\item cura la tenuta dei libri sociali di sua competenza;
\item è responsabile degli adempimenti connessi all’iscrizione nel Runts;
\item curare la gestione dei beni mobili e immobili dell’Associazione o da essa detenuti;
\item emana i regolamenti interni sottoponendoli alla ratifica dell’assemblea qualora almeno un socio lo richieda.
\end{enumerate}
\item Il mandato dura due anni e si è rieleggibili una sola volta consecutivamente. Il mandato del Direttivo è prorogabile di sei mesi su deliberazione dell’Assemblea straordinaria ed è automaticamente prorogato in caso di assenza di candidati. La proroga complessiva non può superare un anno.
\item Sono eletti nel Consiglio Direttivo i candidati che hanno ottenuto il maggior numero di voti fino all’esaurimento dei seggi da assegnare. Si possono candidare tutti gli associati senza alcuna richiesta di sostegno da parte di altri soci.
\item Per l’assegnazione della carica di Presidente, Vicepresidente, Segretario e Tesoriere si procede come di seguito:
\begin{enumerate}
   \item Elezione ordinaria (ossia durante l’Assemblea ordinaria convocata per l’elezione del Direttivo):
   \begin{enumerate}
	\item Ogni candidato al Direttivo, contestualmente alla presentazione della candidatura, può dichiarare due cariche Sociali che vorrebbe ricoprire in ordine di preferenza;
	\item Le cariche vengono assegnate in base al numero di voti raccolti. Al candidato viene assegnata la carica maggiormente preferita ancora vacante;
	\item Qualora, dopo la ridistribuzione, vi siano cariche vacanti, il Direttivo le assegna fra i	propri componenti a maggioranza dei tre quarti dei componenti.
   \end{enumerate}
   \item sostituzione (ossia quando, nel corso del mandato, una carica sociale rimanga vacante per qualsiasi motivo): 
   \begin{enumerate}
	\item il Direttivo è convocato entro dieci giorni dalla vacanza per l’elezione, a maggioranza dei tre quarti dei componenti, del sostituto. la sostituzione del Presidente avviene come disposto dall’articolo 15 comma 3 del presente Statuto.
   \end{enumerate}
\end{enumerate}
\item Le cariche di Presidente, Vicepresidente e Segretario sono fra loro incompatibili.
\item I componenti del Consiglio Direttivo devono astenersi dall’agire in conflitto di interessi. Verificandosi tale caso sono tenuti ad avvisare il Consiglio, l’Assemblea e il Collegio dei Probiviri astenendosi dall’esercitare il diritto di voto.
\item Le riunioni del Direttivo sono valide quando è presente la maggioranza assoluta dei componenti. Il Direttivo delibera a maggioranza dei voti dei presenti. In caso di parità prevale il voto del Presidente.
\item Ogni membro del Direttivo ha la facoltà di delegare parte o la totalità della sua carica ad un membro terzo o ad un Socio, sotto sua supervisione, nel rispetto della normativa vigente, pur conservando la titolarità e la responsabilità morale e legale del proprio mandato. Ogni delega va notificata al resto del direttivo.
\item Nel caso di dimissioni del Consiglio Direttivo, durante il periodo intercorrente fra tali dimissioni e la nomina del nuovo Consiglio Direttivo, il Consiglio dimissionario resta in carica per il disbrigo degli affari di ordinaria amministrazione. Si considera dimissionario l’intero Consiglio Direttivo qualora siano dimissionari almeno la metà più uno dei Consiglieri come eletti originariamente dall’Assemblea.
\item In caso di dimissioni, decesso, decadenza, o altro impedimento di uno o più dei suoi membri subentreranno i Soci che hanno riportato il maggior numero di voti dopo l’ultimo eletto nelle elezioni del Consiglio. A parità di voti la nomina spetta al Socio che ha la maggiore anzianità di iscrizione. Chi subentra in luogo del Consigliere cessato dura in carica per lo stesso periodo residuo del mandato di quest’ultimo.
\item Le dimissioni dalla carica sociale ricoperta e/o dal Direttivo sono presentate al Presidente del Direttivo, al Vicepresidente, al Segretario del Direttivo e al Collegio dei Probiviri e non è richiesta accettazione. Il dimissionario può richiedere la convocazione di un’Assemblea ordinaria allegandone le motivazioni. L’Assemblea, in tal caso, può respingere le dimissioni deliberando sull’argomento sottoposto dal dimissionario.
\item Il Direttivo decade anche a causa di altri motivi individuati nel regolamento interno. Il regolamento stesso prevede anche le modalità di elezione del Direttivo.
\end{enumerate}

\section{Presidenza}
\begin{enumerate}
 \item Il Presidente rappresenta legalmente PoliNetwork, presiede e convoca il Direttivo.
 \item Il Presidente svolge l’ordinaria amministrazione sulla base delle direttive di Assemblea e Direttivo, riferendo a entrambi in merito all’attività compiuta.
 \item Il Presidente cessa per scadenza del mandato, per dimissioni volontarie o per eventuale revoca decisa dall’Assemblea, con la maggioranza dei presenti.
 \item Almeno un mese prima della scadenza del mandato, il Presidente convoca l’Assemblea per l’elezione del nuovo Direttivo.
\end{enumerate}

\section{Vicepresidenza}
\begin{enumerate}
 \item Il Vicepresidente sostituisce il Presidente in ogni sua attribuzione ogniqualvolta questi sia impossibilitato nell’esercizio delle sue funzioni. In caso anch’egli sia impossibilitato viene sostituito dal membro del Direttivo più anziano per iscrizione. In caso di Direttivo dimissionario sarà il Socio più anziano per iscrizione.
 \item Il Vicepresidente ha facoltà di convocare l’Assemblea dei Soci qualora dovesse ritenere il Presidente inadempiente rispetto alle sue funzioni.
 \item Nel caso in cui il Presidente cessi dalle sue funzioni, il Vicepresidente diventerà Presidente e il Direttivo procederà all’elezione del Vicepresidente come previsto dal presente Statuto.
\end{enumerate}

\section{Tesoriere}
\begin{enumerate}
 \item Gestisce le finanze e il patrimonio dell’Associazione, stabilendo la copertura finanziaria delle attività e decidendo le modalità di mantenimento e investimento del patrimonio, in accordo con l’Assemblea dei Soci;
 \item Si occupa di ricercare e gestire i fondi e il supporto economico per realizzare le attività dell’Associazione;
 \item Si occupa di gestire la parte economica delle iscrizioni;
\end{enumerate}

\section{Segretario}
\begin{enumerate}
 \item Il Segretario dirige gli uffici dell’Associazione, cura il disbrigo degli affari ordinari, svolge ogni altro compito a lui demandato dalla presidenza del Consiglio Direttivo dalla quale riceve direttive per lo svolgimento dei suoi compiti; in particolare redige i verbali dell’Assemblea dei Soci e del Consiglio Direttivo, attende alla corrispondenza, cura la tenuta del libro dei Soci, trasmette gli invii per le adunanze dell’Assemblea, provvede ai rapporti tra l’Associazione e le pubbliche amministrazioni, gli enti locali, gli istituti di credito e gli altri enti in genere.
\end{enumerate}

\section{Collegio dei Probiviri}
\begin{enumerate}
 \item La funzione dell’organo, il suo funzionamento, la sua specifica composizione e il suo regolamento elettorale è stabilito da apposito regolamento interno.
 \item I componenti del Collegio dei Probiviri devono astenersi dall’agire in conflitto di interessi. Verificandosi tale caso sono tenuti ad avvisare il Collegio, l’Assemblea e il Direttivo astenendosi dall’esercitare il diritto di voto.
 \item Le riunioni del Collegio sono valide quando è presente la maggioranza assoluta dei componenti. Il Collegio delibera a maggioranza dei voti dei presenti.
 \item I componenti del Collegio sono nominati anche tra i non Soci e dovranno essere scelti,
 preferibilmente, in quanto dotati di adeguata esperienza in campo giuridico e/o legale.
 Il Collegio ha facoltà di opporsi a qualsiasi decisione all’interno dell’associazione se ritiene che questa sia in conflitto con il presente Statuto, il regolamento interno, la normativa nazionale o comunitaria o qualsiasi altra forma di regolamentazione, nelle forme e nei limiti descritti nel regolamento interno.
\end{enumerate}

\section{Bilancio e Patrimonio}
\begin{enumerate}
 \item I documenti di bilancio dell’organizzazione sono annuali e decorrono dal primo gennaio di ogni anno. Sono redatti ai sensi degli articoli 13, 14 e 87 del D. Lgs. 117/2017 e delle relative norme di attuazione;
 \item Il bilancio è predisposto dal Consiglio Direttivo e viene approvato dall’Assemblea ordinaria entro 6 (sei) mesi dalla chiusura dell’esercizio cui si riferisce il consuntivo e depositato presso il Registro Unico Nazionale del Terzo Settore entro il 30 (trenta) giugno di ogni anno;
 \item L’organizzazione ha il divieto di distribuire, anche in modo indiretto, utili e avanzi di gestione nonché fondi, riserve o capitale durante la propria vita, ai sensi dell’art. 8 comma 2 del D.Lgs. 117/2017, nonché l’obbligo di utilizzare il patrimonio, comprensivo di eventuali ricavi, rendite, proventi, entrate comunque denominate, per lo svolgimento dell’attività statutaria ai fini dell’esclusivo perseguimento delle finalità previste.
 \item L’ente si dota del revisore contabile e dell’organo di controllo nei casi previsti per legge con specifica delibera assembleare, congiuntamente all’approvazione del regolamento di funzionamento degli organi;
 
\end{enumerate}

\section{Scioglimento e devoluzione del Patrimonio}
\begin{enumerate}
 \item In caso di estinzione o scioglimento, il patrimonio residuo è devoluto, salva diversa destinazione imposta dalla legge, ad altri enti del Terzo Settore, secondo quanto previsto dall’art. 9 del D.Lgs 117/2017.
\end{enumerate}

\section{Risorse economiche}
\begin{enumerate}
 \item Le risorse economiche dell'associazione sono costituite da:
 \vspace{-5pt}
 \begin{enumerate}
  \item quote associative;
  \item donazioni;
  \item lasciti testamentari;
  \item rendite patrimoniali;
  \item attività di raccolta fondi;
  \item rimborsi da convenzioni;
  \item ogni altra entrata ammessa ai sensi del D.Lgs 117/2017 e dalla normativa vigente.
 \end{enumerate}
\end{enumerate}


\section{Trasparenza e Libri Sociali}
\begin{enumerate}
 \item L'Associazione deve tenere i seguenti libri: 
 \vspace{-5pt}
 \begin{enumerate}
  \item libro dei Soci;
  \item libro delle adunanze e delle deliberazioni dell'Assemblea dei Soci;
  \item libro delle adunanze e delle deliberazioni del Consiglio Direttivo;
  \item libro delle adunanze e delle deliberazioni del Collegio dei Probiviri, se nominato.
 \end{enumerate}
 
 \item I libri sociali sono tenuti dall'organo a cui si riferiscono presso la sede legale e/o in formato digitale su piattaforma di archiviazione remota che rispetta la normativa comunitaria e nazionale in materia di gestione dei dati personali e preventivamente comunicata all’Assemblea, in libera visione a tutti i Soci. In essi sono trascritti i verbali delle riunioni, inclusi quelli redatti per atto pubblico.
\end{enumerate}

\section{Norme di rinvio e disposizioni finali}
\begin{enumerate}
 \item Per quanto non è previsto dal presente Statuto, si fa riferimento alle normative vigenti in materia, con particolare riferimento al D. Lgs. 117/2017, ed ai principi generali dell’ordinamento giuridico.
\end{enumerate}
}
\end{document}
