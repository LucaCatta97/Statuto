\documentclass[legalpaper, 11pt]{exam}


\usepackage{hyperref}

\hypersetup
{
  colorlinks,
  citecolor=black,
  linkcolor=black,
  urlcolor=black
}

\usepackage{color,soul}

\usepackage{titlesec}
\titlespacing*{\section}{0pt}{0.5\baselineskip}{0.5\baselineskip}

\tolerance=1
\emergencystretch=\maxdimen
\hyphenpenalty=10000
\hbadness=10000

\usepackage[utf8]{inputenc}
\usepackage[left=0.5in,
            right=0.5in,
            top=0.85in,
            bottom=1in]{geometry}
\usepackage{lastpage}

\cfoot{}
\rfoot{Pagina {\thepage} di \pageref{LastPage}}
\rhead{\textmd{\textsf{PoliNetwork - Statuto}}}

\let\tempone\enumerate
\let\temptwo\endenumerate
\renewenvironment{enumerate}{\tempone\addtolength{\itemsep}{-0.45\baselineskip}}{\temptwo}

%\usepackage{fontspec}
%\defaultfontfeatures{Mapping=tex-text}
%\setmainfont{Verdana}


%\graphicspath{{assets/}}

%\iffalse
\usepackage{fontspec}
\setmainfont{Ubuntu} [
    Path = fonts/Ubuntu/,
    Extension = .ttf,
    UprightFont = *-Regular,
    ItalicFont = *-Italic,
    BoldFont = *-Bold,
    BoldItalicFont = *-BoldItalic
]
%\fi


%\title{PoliNetwork \\ Statuto}
%\author{PoliNetwork}
%\date{\today}


%\addto\captionsenglish{% 
%  \renewcommand{\contentsname}%
%    {Indice}%
%}

\usepackage{setspace}


\begin{document}

{
\setstretch{0.5}

\begin{center}
\fbox{\fbox{\parbox{4in}{

\begin{center}
{\textbf{Statuto PoliNetwork}}

\end{center}

}}}

\end{center}


\section{Denominazione}
\begin{enumerate}
 \item È costituito, nel rispetto del Codice civile, del D. Lgs. 117/2017 e della normativa in materia, l’Ente del Terzo Settore denominato: associazione studentesca “PoliNetwork APS”, di seguito denominata “PoliNetwork” e formata da studenti del Politecnico di Milano.
 \item PoliNetwork assume la forma giuridica di associazione non riconosciuta, apartitica e aconfessionale.
 \item L'associazione ha durata illimitata.
\end{enumerate}

\section{Definizioni}
\begin{enumerate}
 \item Si definisce PoliNetwork anche come "associazione" oppure "organizzazione".
 \item Si definisce "limite temporale" il trascorrere di 30 (trenta) giorni, 90 (novanta) giorni se la richiesta è presentata a partire da un mese prima dell'inizio delle sessioni d'esame e fino alla fine delle stesse.
 \item Si definisce "canale di contatto" qualsiasi mezzo di comunicazione scritta definito nel regolamento interno dell'associazione e utilizzato per qualsiasi richiesta a cui si fa riferimento nel presente Statuto, purché gestito nel rispetto della normativa nazionale e comunitaria in materia di protezione dei dati personali.
\end{enumerate}

\section{Sede}
\begin{enumerate}
 \item L’associazione studentesca ha sede legale presso l’indirizzo di residenza del Socio individuato dal regolamento interno e comunicata ai sensi della normativa vigente.
 \item Il trasferimento della sede legale non comporta modifica statutaria, ma l’obbligo di comunicazione agli uffici competenti.
\end{enumerate}

\section{Statuto}
\begin{enumerate}
 \item L’associazione studentesca è disciplinata dal presente Statuto, ed agisce nei limiti del D. Lgs. 117/2017, delle relative norme di attuazione, della legge regionale e dei principi generali dell’ordinamento giuridico.
 \item Lo Statuto vincola alla sua osservanza gli aderenti all’associazione; esso costituisce le regole fondamentali di comportamento e l’organizzazione della stessa.
\end{enumerate}

\section{Scopo, finalità e attività}
L’Associazione: 
\vspace{-5pt}
\begin{enumerate}
 \item Non ha scopo di lucro ed opera in ambito universitario;
 \item Promuove iniziative culturali, sociali e di aggregazione dedicate alla popolazione studentesca del Politecnico di Milano;
 \item È libera da ogni condizionamento politico, economico e religioso ed opera nel rispetto del codice etico del Politecnico di Milano;
 \item Realizza e gestisce progetti di pubblica utilità di carattere tecnico informatico con licenza codice aperto;
 \item Realizza e gestisce spazi per la comunicazione tra gli studenti del Politecnico di Milano;
 \item Facilita la comunicazione tra la popolazione studentesca del Politecnico di Milano e le altre figure del Politecnico di Milano.
\end{enumerate}

\section{Organi}
Sono organi dell’Associazione: 
\vspace{-5pt}
\begin{enumerate}
 \item il Consiglio Direttivo (di seguito anche Direttivo);
 \item le Assemblee dei Soci ordinaria e straordinaria (di seguito anche Assemblee);
 \item Il Collegio dei Probiviri.
\end{enumerate}

\section{Associati}
\begin{enumerate}
 \item Il numero dei Soci è illimitato. 
 \item Possono essere Soci dell’associazione solo ed esclusivamente gli studenti che ne fanno richiesta, regolarmente iscritti ai corsi di studi di primo o secondo livello o di dottorato del Politecnico di Milano di tutte le scuole.
 \item Gli studenti sono ammessi senza alcuna forma di discriminazione di sesso, identità di genere, etnia, opinione politica, orientamento sessuale e religione.
 \item Essi devono condividere gli scopi dell’associazione.
\end{enumerate}

\section{Volontari}
\begin{enumerate}
 \item Le attività perpetuate dai Soci ai fini del raggiungimento degli scopi associativi sono svolte puramente a scopo di volontariato e non condizionate ad alcun compenso.
 \item L'associazione può avvalersi nello svolgimento delle proprie attività dell'opera di volontari. Sono volontari coloro che per libera scelta, senza divenire associati, svolgono attività in favore di PoliNetwork o dei suoi progetti mettendo a disposizione il proprio tempo e le proprie capacità in modo personale e gratuito senza fini di lucro. I volontari possono essere iscritti in apposito Registro dei Volontari.
\end{enumerate}

\section{Ammissione}
\begin{enumerate}
 \item Chi intende essere ammesso come Socio dovrà presentare la relativa richiesta al Consiglio Direttivo, impegnandosi ad attenersi al presente Statuto, ad osservarne gli eventuali regolamenti interni e le delibere adottate dagli organi dell’associazione e a mantenere attivo il canale di contatto per tutta la durata del rapporto associativo. Il Consiglio Direttivo, o disgiuntamente ogni suo componente o delegato, potrà chiedere all’aspirante Socio ogni documentazione utile ed avvalersi di qualsiasi mezzo ritenga necessario al fine di valutare la richiesta di ammissione, nel rispetto della normativa vigente, sia comunitaria sia nazionale, anche in tema di tutela dei dati personali e della privacy.
 \item Il Consiglio Direttivo deciderà, entro il limite temporale, sull’esito positivo o meno della richiesta. Nel caso non pervenga alcuna risposta la richiesta è da considerarsi respinta. La delibera di rigetto della domanda di ammissione dovrà essere motivata all’Assemblea dei Soci e all'aspirante Socio, entro ulteriori 30 (trenta) giorni, qualora richiesto da quest'ultimo o da un Socio.
 \item All’atto dell’accettazione della richiesta da parte dell’associazione il richiedente dovrà versare la quota associativa e successivamente acquisirà ad ogni effetto la qualifica di Socio.
 \item Tutti i Soci verranno annotati nel libro dei Soci dell’associazione.
\end{enumerate}

\section{Diritti e doveri degli associati}
\begin{enumerate}
 \item Gli Associati dell'organizzazione hanno il diritto di:
 \vspace{-5pt}
 \begin{enumerate}
  \item eleggere gli organi sociali e di essere eletti negli stessi, secondo il presente Statuto;
  \item essere informati sulle attività dell’organizzazione e controllarne l’andamento;
  \item prendere atto dell’ordine del giorno delle assemblee, prendere visione del rendiconto economico-finanziario, consultare i verbali;
  \item votare in Assemblea. Ciascun associato ha diritto al più ad un voto.
 \end{enumerate}
\item Gli stessi Associati hanno il dovere di:
\vspace{-5pt}
\begin{enumerate}
 \item rispettare il presente Statuto e il regolamento interno;
 \item astenersi da qualsiasi atto che possa nuocere all’organizzazione;
 \item versare la quota associativa secondo l’importo stabilito. La quota associativa è personale, non è rimborsabile e non può essere trasferita a terzi o rivalutata.
\end{enumerate}

\end{enumerate}

\section{Recesso}
\begin{enumerate}
 \item La qualità di Socio si perde per:
 \vspace{-5pt}
 \begin{enumerate}
  \item mancato pagamento della quota sociale: la decadenza avviene su decisione dell’Assemblea, previa proposta del Consiglio Direttivo, trascorso un mese dal mancato versamento della quota sociale;
  \item dimissioni: ogni Socio può recedere dall’associazione in qualsiasi momento dandone comunicazione scritta al Consiglio Direttivo; tale recesso avrà decorrenza immediata. Resta fermo l’obbligo di pagamento della quota sociale per l’anno in corso;
  \item espulsione: il Consiglio Direttivo delibera l’espulsione del Socio, previa contestazione degli eventuali addebiti e, se possibile, sentito il Socio interessato, per atti compiuti in contrasto a quanto previsto dal presente statuto o dal regolamento interno o qualora siano intervenuti gravi motivi che rendano incompatibile la prosecuzione del rapporto associativo. Il Consiglio Direttivo comunica la decisione all’Assemblea e al Collegio dei Probiviri, che hanno, anche disgiuntamente, facoltà di opporsi alla decisione, sempre entro il limite temporale. Qualora il Collegio dei Probiviri si opponga alla decisione dell'Assemblea, quest'ultima è tenuta ad ascoltare il Collegio nella prima seduta disponibile per confermare o annullare la propria decisione. L'espulsione in questo caso prende effetto dopo il voto di conferma;
  \item perdita dello status di studente;
  \item decesso.
 \end{enumerate}

 \item Gli Associati che abbiano comunque cessato di appartenere all’associazione non possono richiedere i contributi versati e non hanno alcun diritto sul patrimonio dell’associazione stessa.
 \item Chiunque perda o non abbia mai ottenuto la qualità di Socio e abbia dato un contributo all’associazione può richiedere al Direttivo di essere inserito nel libro degli alumni di Polinetwork. Se entro il limite temporale non è stata ricevuta risposta la richiesta è da intendersi respinta. I fondatori sono inseriti di diritto. Le disposizioni del presente comma si applicano anche a coloro che abbiano partecipato alle attività di PoliNetwork prima dell’emanazione del presente statuto e che abbiano perso lo status di studente. L'Assemblea può in qualsiasi momento rimuovere chiunque dal libro degli alumni.
\end{enumerate}

\section{Assemblea}
\begin{enumerate}
 \item L’Assemblea è l’organo sovrano di PoliNetwork ed è composta da tutti i Soci dell’organizzazione.
 \item L’Assemblea è presieduta e convocata almeno una volta all’anno dal Presidente o dal Vicepresidente dell’organizzazione o da chi ne fa le veci mediante avviso scritto da inviare almeno 2 (due) giorni prima di quello fissato per l’adunanza e contenente la data della riunione, l’orario, il luogo (anche virtuale), l’ordine del giorno e l’eventuale data di seconda convocazione. Tale comunicazione avviene per mezzo del canale di contatto.
 \item L’Assemblea viene convocata per l’approvazione del bilancio o del rendiconto e quando ne è fatta richiesta motivata da almeno un quinto dei Soci o quando il Collegio dei Probiviri o il Consiglio Direttivo o disgiuntamente ogni suo componente o delegato lo ritiene necessario.
 \item Delle riunioni dell’Assemblea è redatto il verbale, sottoscritto dal Presidente e dal verbalizzante.
 \item L’Assemblea ordinaria delibera su tutti e soli i punti elencati al comma successivo del presente Statuto. L’Assemblea straordinaria viene proposta ogni qual volta sia necessario per le esigenze dell’associazione.
 \item L'Assemblea:
 \vspace{-5pt}
 \begin{enumerate}
  \item approva il bilancio;
  \item determina le linee generali programmatiche riguardo l’attività dell’organizzazione;
  \item elegge e revoca i componenti del Consiglio Direttivo;
  \item elegge e revoca, quando previsto, il Collegio dei Probiviri su proposta del Direttivo o dell’Assemblea stessa nel rispetto dello specifico regolamento di funzionamento;
  \item delibera la responsabilità dei componenti degli organi sociali e promuove azione di responsabilità nei loro confronti;
  \item delibera le modificazioni dell’Atto Costitutivo o dello Statuto;
  \item approva l’eventuale regolamento dei lavori assembleari;
  \item delibera lo scioglimento, la trasformazione, la fusione o la scissione dell’organizzazione;
  \item delibera sull’esclusione dei Soci.
 \end{enumerate}
 \item Le Assemblee ordinaria e straordinaria sono regolarmente costituite in prima convocazione con la presenza di più del 50\% degli aderenti, presenti in proprio o per delega, e in seconda convocazione, da tenersi anche nello stesso giorno, qualunque sia il numero degli aderenti presenti, in proprio o per delega.
 \item L’Assemblea delibera a maggioranza dei voti dei presenti, in caso di parità prevale il voto del Presidente.
 \item Gli Associati possono farsi rappresentare in Assemblea solo da altri Associati, conferendo delega scritta tramite canale di contatto. Ciascun Associato è portatore di al più una delega.
 \item Le delibere sono espresse con voto palese, fatta eccezione per quelle riguardanti singoli o ristretti gruppi di individui o quando l’Assemblea, il Direttivo o il Socio firmatario della proposta lo ritenga opportuno o se prescritto dal regolamento interno, le quali sono espresse con voto segreto;
 \item Per le decisioni circa Statuto, Atto Costitutivo, nomina del Direttivo, regolamento interno e organizzazione di attività sono considerati aventi diritto di voto i soli soci con una partecipazione attiva nell’organizzazione delle attività, tali Soci prendono il nome di Soci attivi, per la cui descrizione esaustiva si rimanda al regolamento interno.
\end{enumerate}

\section{Direttivo}
\begin{enumerate}
 \item Il Direttivo è composto da un minimo di 5 (cinque) a un massimo di 9 (nove) persone, sempre in numero dispari. Le cariche vengono specificate nel regolamento interno, ad eccezione di Presidente, Vicepresidente, Tesoriere e Segretario.
 \item Le modalità di attribuzione e la durata del mandato vengono specificate nel regolamento interno.
 \item I componenti il Consiglio Direttivo devono astenersi dall’agire in conflitto di interessi. Verificandosi tale caso sono tenuti ad avvisare il Consiglio astenendosi dall’esercitare il diritto di voto.
 \item Le riunioni del Direttivo sono valide quando è presente la maggioranza assoluta dei componenti. Il Direttivo delibera a maggioranza dei voti dei presenti.
 \item Ogni membro del Direttivo ha la facoltà di delegare parte o la totalità della sua carica ad un membro terzo o ad un Socio, sotto sua supervisione, nel rispetto della normativa vigente, pur conservando la titolarità e la responsabilità morale e legale del proprio mandato.
 \item Nel caso di dimissioni del Consiglio Direttivo, durante il periodo intercorrente fra tali dimissioni e la nomina del nuovo Consiglio Direttivo, il Consiglio dimissionario resta in carica per il disbrigo degli affari di ordinaria amministrazione. Si considera dimissionario l’intero Consiglio Direttivo qualora siano dimissionari almeno la metà più uno dei Consiglieri come eletti originariamente dall’Assemblea.
 \item In caso di dimissioni, decesso, decadenza, o altro impedimento di uno o più dei suoi membri subentreranno i Soci che hanno riportato il maggior numero di voti dopo l’ultimo eletto nelle elezioni del Consiglio. A parità di voti la nomina spetta al Socio che ha la maggiore anzianità di iscrizione. Chi subentra in luogo del Consigliere cessato dura in carica per lo stesso periodo residuo del mandato di quest'ultimo.
 \item Nel caso di dimissioni di un membro del Direttivo per il quale non possa esserci subentro, il Direttivo nominerà entro il limite temporale un membro sostitutivo per cooptazione. L'Assemblea può opporsi, convocando nuove elezioni per la sola carica mancante o per l'intero Direttivo.
 \item Il Direttivo decade anche a causa di altri motivi individuati nel regolamento interno. Il regolamento stesso prevede anche le modalità di elezione del Direttivo.
\end{enumerate}

\section{Presidenza}
\begin{enumerate}
 \item Il Presidente rappresenta legalmente PoliNetwork, presiede e convoca il Direttivo.
 \item Il Presidente svolge l’ordinaria amministrazione sulla base delle direttive di Assemblea e Direttivo, riferendo a entrambi in merito all’attività compiuta.
 \item Il Presidente cessa per scadenza del mandato, per dimissioni volontarie o per eventuale revoca decisa dall’Assemblea, con la maggioranza dei presenti.
 \item Almeno un mese prima della scadenza del mandato, il Presidente convoca l’Assemblea per l’elezione del nuovo Direttivo.
\end{enumerate}

\section{Vicepresidenza}
\begin{enumerate}
 \item Il Vicepresidente sostituisce il Presidente in ogni sua attribuzione ogniqualvolta questi sia impossibilitato nell’esercizio delle sue funzioni. In caso anch'egli sia impossibilitato viene sostituito dal membro del Direttivo più anziano per iscrizione. In caso di Direttivo dimissionario sarà il Socio più anziano per iscrizione.
 \item Il Vicepresidente ha facoltà di convocare l’Assemblea dei Soci qualora dovesse ritenere il Presidente inadempiente rispetto alle sue funzioni.
\end{enumerate}

\section{Tesoriere}
\begin{enumerate}
 \item Gestisce le finanze e il patrimonio dell’Associazione, stabilendo la copertura finanziaria delle attività e decidendo le modalità di mantenimento e investimento del patrimonio, in accordo con l’Assemblea dei Soci;
 \item Si occupa di ricercare e gestire i fondi e il supporto economico per realizzare le attività dell’Associazione;
 \item Si occupa di gestire le iscrizioni;
\end{enumerate}

\section{Segretario}
\begin{enumerate}
 \item Il Segretario dirige gli uffici dell’Associazione, cura il disbrigo degli affari ordinari, svolge ogni altro compito a lui demandato dalla presidenza del Consiglio Direttivo dalla quale riceve direttive per lo svolgimento dei suoi compiti; in particolare redige i verbali dell’Assemblea dei Soci e del Consiglio Direttivo, attende alla corrispondenza, cura la tenuta del libro dei Soci, trasmette gli invii per le adunanze dell’Assemblea, provvede ai rapporti tra l’Associazione e le pubbliche amministrazioni, gli enti locali, gli istituti di credito e gli altri enti in genere.
\end{enumerate}

\section{Collegio dei Probiviri}
\begin{enumerate}
 \item Il Collegio dei Probiviri viene chiamato a giudicare su eventuali divergenze o questioni nate all’interno dell’organizzazione, sulle violazioni dello statuto (e/o dei regolamenti interni), sull’inosservanza delle delibere e sull’esclusione dei Soci.
 \item La funzione dell’organo, il suo funzionamento, la sua specifica composizione e il suo regolamento elettorale è stabilito da apposito regolamento interno.
 \item I componenti del Collegio dei Probiviri sono nominati anche tra i non Soci e dovranno essere scelti in quanto dotati di adeguata esperienza in campo giuridico e/o legale.
 \item Il Collegio dei Probiviri ha facoltà di opporsi a qualsiasi decisione all'interno dell'associazione se ritiene che questa sia in conflitto con il presente Statuto, il regolamento interno, la normativa nazionale o comunitaria o qualsiasi altra forma di regolamentazione. In questo caso, l'Assemblea è tenuta ad ascoltare il Collegio nella prima seduta disponibile per confermare o annullare la propria decisione. La decisione in questo caso prende effetto dopo il voto di conferma.
\end{enumerate}

\section{Bilancio e Patrimonio}
\begin{enumerate}
 \item I documenti di bilancio dell’organizzazione sono annuali e decorrono dal primo gennaio di ogni anno. Sono redatti ai sensi degli articoli 13, 14 e 87 del D. Lgs. 117/2017 e delle relative norme di attuazione;
 \item Il bilancio è predisposto dal Consiglio Direttivo e viene approvato dall’Assemblea ordinaria entro 6 (sei) mesi dalla chiusura dell’esercizio cui si riferisce il consuntivo e depositato presso il Registro Unico Nazionale del Terzo Settore entro il 30 (trenta) giugno di ogni anno;
 \item L’organizzazione ha il divieto di distribuire, anche in modo indiretto, utili e avanzi di gestione nonché fondi, riserve o capitale durante la propria vita, ai sensi dell’art. 8 comma 2 del D.Lgs. 117/2017, nonché l’obbligo di utilizzare il patrimonio, comprensivo di eventuali ricavi, rendite, proventi, entrate comunque denominate, per lo svolgimento dell’attività statutaria ai fini dell’esclusivo perseguimento delle finalità previste.
 \item L’ente si dota del revisore contabile e dell’organo di controllo nei casi previsti per legge con specifica delibera assembleare, congiuntamente all’approvazione del regolamento di funzionamento degli organi;
 
\end{enumerate}

\section{Scioglimento e devoluzione del Patrimonio}
\begin{enumerate}
 \item In caso di estinzione o scioglimento, il patrimonio residuo è devoluto, salva diversa destinazione imposta dalla legge, ad altri enti del Terzo Settore, secondo quanto previsto dall’art. 9 del D.Lgs 117/2017.
\end{enumerate}

\section{Risorse economiche}
\begin{enumerate}
 \item Le risorse economiche dell'associazione sono costituite da:
 \vspace{-5pt}
 \begin{enumerate}
  \item quote associative;
  \item donazioni;
  \item lasciti testamentari;
  \item rendite patrimoniali;
  \item attività di raccolta fondi;
  \item rimborsi da convenzioni;
  \item ogni altra entrata ammessa ai sensi del D.Lgs 117/2017 e dalla normativa vigente.
 \end{enumerate}
\end{enumerate}


\section{Trasparenza e Libri Sociali}
\begin{enumerate}
 \item L'Associazione deve tenere i seguenti libri: 
 \vspace{-5pt}
 \begin{enumerate}
  \item libro dei Soci;
  \item libro delle adunanze e delle deliberazioni dell'Assemblea dei Soci;
  \item libro delle adunanze e delle deliberazioni del Consiglio Direttivo;
  \item libro delle adunanze e delle deliberazioni del Collegio dei Probiviri, se nominato.
 \end{enumerate}
 
 \item I libri sociali sono tenuti dall'organo a cui si riferiscono presso la sede legale e/o in formato digitale su piattaforma di archiviazione remota che rispetta la normativa comunitaria e nazionale in materia di gestione dei dati personali e preventivamente comunicata all’Assemblea, in libera visione a tutti i Soci. In essi sono trascritti i verbali delle riunioni, inclusi quelli redatti per atto pubblico.
\end{enumerate}

\section{Modifica Statuto, Atto Costitutivo e dissoluzione ente}
\begin{enumerate}
\item La modifica dello Statuto dell’organizzazione o dell’Atto Costitutivo e lo scioglimento con conseguente liquidazione nonché devoluzione del patrimonio sono deliberabili da Assemblea straordinaria convocata dal consiglio Direttivo. La Delibera è approvata e ritenuta valida con il voto favorevole di almeno 2/3 (due terzi) degli aventi diritto di voto.
\end{enumerate}

\section{Norme di rinvio e disposizioni finali}
\begin{enumerate}
 \item Per quanto non è previsto dal presente Statuto, si fa riferimento alle normative vigenti in materia, con particolare riferimento al D. Lgs. 117/2017, ed ai principi generali dell’ordinamento giuridico.
\end{enumerate}
}
\end{document}
